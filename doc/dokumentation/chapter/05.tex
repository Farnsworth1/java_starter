%!TEX root = ./main.tex
%
% This file is part of the i10 thesis template developed and used by the
% Media Computing Group at RWTH Aachen University.
% The current version of this template can be obtained at
% <https://hci.rwth-aachen.de/karrer_thesistemplate>.

\chapter{Benutzeranleitung}

\section{Verzeichnisstuktur}

\begin{itemize}
    \item README.md ist die Readme-Datei für das Projekt
    \item \texttt{src} : In diesem Verzeichnis befinden sich alle Testklassen, sowie der Quellcode des Programms
    \item \texttt{out}: In diesem Verzeichnis befinden sich alle kompilierten Java Dateien
    \item \texttt{doc} : In diesem Verzeichnis befindet sich die Programm-Dokumentation
    \begin{itemize}
        \item \texttt{javadoc}: In diesem Verzeichnis liegt die generierte Entwicklerdokumentation
        \item \texttt{documentation}: In diesem Verzeichnis liegen die Latex-Dateien zur Erzeugung dieser Dokumentation
        \item \texttt{Dokumentation.pdf}: Dieses Dokument
    \end{itemize}
    \item \texttt{tests}: In diesem Verzeichnis befinden sich Beispiele für die Eingabe-, Ausgabe und Fehlerdateien
    \begin{itemize}
        \item \texttt{ihk} für Beispiele aus der Aufgabenstellung
        \item \texttt{itests} für selbst ausgesuchte Beispiele 
    \end{itemize}
\end{itemize}

\section{Systemanforderungen}

Die Software ist unter Java JRE 11 (Java Runtime Environment Version 11) ausführbar. Dafür muss die genannte Java-Version oder höher auf der Maschine installiert sein. Außerdem sollte beachtet werden, dass Java nicht abwährtskompatibel ist und entsprechend keine ältere Version verwendet werden sollte.\\
Das Programm wurde wie im Kapitel \ref{chap:Env} beschrieben entwickelt.\\
Die Ausführbarkeit wird unter dem genannten Voraussetzungen garantiert.\\
Zudem kann das Programm unter Linux/Unix mit oben gennanten Vorausetzungen (Java-Version) ausgeführt werden.



\section{Benutzung der Kommandozeile unter Windows}


Mit Hilfe der Kommandozeile (Eingabeaufforderung unter Windows bzw. Shell unter Linux/Unix) müssen Sie in das Verzeichnis \texttt{\texttildelow \textbackslash out\textbackslash artifacts\textbackslash makeUpper\textunderscore jar\textbackslash}. 
Das Programm wird wie folgt gestartet: 

Beispielaufruf: \texttt{java -jar T9.jar} falls das Programm in Kommandozeile-Modus aufgerufen werden sollte

Beispielaufruf: \texttt{java -jar T9.jar -i testfall.txt} falls das Programm mit einer Eingabedatei testfall.txt ausgeführt werden sollte.


\section{Die Ausführung der Testfälle}
Je nach Betriebssystem liegt ein Bash- oder Shell-Script zur Verfügung, um die Testfälle im Verzeichnis \textit{tests} automatisiert auszuführen.


Die Dateien (run\textunderscore tests.bat bzw. run\textunderscore test.sh) führen die oben genannten Befehle automatisiert mit allen Testfällen im Verzeichnis \texttt{tests} aus. Dabei muss die Pfadeingabe zur .jar-Datei und der zur Testdatei(en) korrekt sein. 
Für nähere Informationen zu Testfällen siehe Kapitel  \ref{chap:Testing}
