%!TEX root = ./main.tex
%
% This file is part of the i10 thesis template developed and used by the
% Media Computing Group at RWTH Aachen University.
% The current version of this template can be obtained at
% <https://hci.rwth-aachen.de/karrer_thesistemplate>.

\chapter{Entwicklungsumgebung}
\label{chap:Env}

Die Entwicklung und Tests der Software wurden auf einem System mit folgender Spezifikation durchgeführt:
\begin{itemize}
\item Betriebsystem: Microsoft Windows 10 Enterprise (64 Bit)
\item  Prozessor: AMD FX(tm)-8350 Eight-Core Processor, 4000 MHz, 4 Kern(e), 8 logische(r) Prozessor(en)
\item Installierter physischer Speicher (RAM) 16,0 GB
\item Grafikkarte: NVIDIA GeForce 210
\end{itemize}

Weitere Spezifikationen:
\begin{itemize}
\item Die Software wurde mit der Programmiersprache Java in der Version 11.0.2 entwickelt. Verwendet wurde die IDE Intellij IDEA Community 2019.3.2.
\item Zur Erstellung von Textdateien wurde der Texteditor Notepad++ benutzt.
\item Für die Klassen- und Sequenzdiagramm wurde das Programm Visual Pardigm 16 verwendet.
\item Zur Erstellung von Nassi-Schneiderman-Diagrammen wurde Java-Editor18.23 verwendet.
\item Zur Erstellung dieser Dokumentation wurde MiKTeX 2.9 bzw. ein Online-Latex-Editor Overleaf verwendet.
\end{itemize}