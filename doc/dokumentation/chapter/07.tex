%!TEX root = ./main.tex
%
% This file is part of the i10 thesis template developed and used by the
% Media Computing Group at RWTH Aachen University.
% The current version of this template can be obtained at
% <https://hci.rwth-aachen.de/karrer_thesistemplate>.

\chapter{Abweichungen des Prüfungsproduktes vom Konzept}

Bei der Umsetzung des Konzeptes in das Programmsystem wurden einige kleine Änderungen vorgenommen. Folgende Änderungen am Konzept sind anzuführen:

\begin{itemize}
    \item Die Baumstruktur in der Klasse \texttt{Baum} speichert die Wörter mit ihrem expliziten Ziffernfolge. Dies liegt daran, dass die Reihenfolge der Buchstaben in einem Knoten nicht ausreicht, um Wörter eineindeutig zuzuordnen.
    \item Die Klassenstruktur hat noch zusätzliche Hilfsfunktionen wie \texttt{nutzerinteraktion()} in \textbf{Konsole} oder \texttt{makeWord()} in \textbf{Woerterbuch}
    \item Die Nassi-Schneidermann-Diagrammen ersetzen den im Prüfungsproduktes aufgeführten Pseudocode.
\end{itemize}