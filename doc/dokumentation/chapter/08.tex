%!TEX root = ./main.tex
%
% This file is part of the i10 thesis template developed and used by the
% Media Computing Group at RWTH Aachen University.
% The current version of this template can be obtained at
% <https://hci.rwth-aachen.de/karrer_thesistemplate>.

\chapter{Zusammenfassung und Ausblick}

Es kann zahlreiche Erweiterungen zu dieser Software geben. Möglichkeiten wären beispielsweise, eine erweiterte Interpretation der Tasten, um Zahlen darzustellen, Darstellung von Klein und Großschreibung.

Für die Eingabe sowie die Ausgabe könnte eine graphische Oberfläche implementiert werden. Dadurch könnte der Anwender die Ausgabe graphisch besser sehen.

Das Programm speichert die Eingaben von Nutzer im Wörterbuch nur bei einer expliziten Eingabe von Zifferncode. Dies kann einfacher gestaltet werden, indem die Wortvorschläge nicht nur aus dem Wörterbuch vorgeschlagen werden, sondern aus dem Tippverhalten des Nutzers. Ein Wörterbuch bestehend aus Konstellationen von den Häufigsten benutzen Buchstaben können vorgeschlagen werden.